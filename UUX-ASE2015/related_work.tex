\subsection{Mining App Store Reviews}

Harman \etal \cite{paper:Harman} introduced app store mining and analyzed technical and business aspects of apps by extracting app features from the official app descriptions. 

Chandy and Gu \cite{paper:Chandy} classified spam in the AppStore through a latent model capable of classifying apps, developers, reviews and users into the normal and malicious categories. 

Pagano and Maalej \cite{paper:Pagano} investigated the types of user feedback present in the reviews and applied frequent item set mining for identifying feedback type patterns in user reviews, we map some of their findings into the labels we used in this work. 

Iacob and Harrison \cite{paper:Iacob} extracted feature requests from app store reviews by means of linguistic rules and used Latent Dirichlet Allocation (LDA) \cite{paper:Blei2003} to group the feature requests. In contrast with this work, we employed linguistic rules, text analysis, and sentiment analysis to mine different information from user reviews (not only feature requests).  
LDA was also used for: (i) feature based sentiment analysis of reviews\cite{paper:Guzman}, (ii) user reviews summarization\cite{paper:Galvis}, and (iii) the identification of incorrectly rated reviews\cite{paper:Fu}. 
%Guzman and Maalej\cite{paper:Guzman} extracted app features through a collocation and LDA algorithm. They assigned each %feature a sentiment through sentiment analysis. %Our work could be complemented with this existing work so that the extracted %features are associated to a specific software maintenance cateogry. 
%Galvis Carre\~no and Winbladh \cite{paper:Galvis} applied LDA to summarize user reviews. Our approach could use the topics %generated by LDA to group semantically similar reviews which belong to the same category. 
%Fu \etal \cite{paper:Fu} analyze user reviews from Google Play and apply a linear regression model combining the text from the %reviews and its ratings to identify incorrectly rated reviews. They input the words classified as negative words into an LDA %algorithm to find the main reason why users are unsatisfied with the app. 

Chen \etal \cite{paper:ARminer} used Naive Bayes for finding informative review sentences and LDA for grouping sentences with similar content. They then rank the groups of reviews according to a scheme which analyzes volume, time patterns and ratings. In our evaluation we filtered non-informative reviews using Chen's \etal approach. Similarly to Chen \etal \cite{paper:ARminer} we could rank the sentences that are considered more important in each of the software maintenance and evolution categories. 

Li \etal \cite{paper:Li} analyze user reviews to measure user satisfaction by matching words or phrases in the user comments with a predefined dictionary. 

\subsection{Classifying Reviews in UUX Dimensions} 

