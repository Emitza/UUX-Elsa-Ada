\noindent
Usability and User Experience (UUX) play an increasingly significant role in the success of software products. There exist an extensive number of UUX evaluation methods. However, these methods are expensive in terms of time and human resources. Additionally, they capture the interaction of users with the system for a short time, in a non-natural environment. Users express their opinion about software products on review sites, social media platforms and blogs. They write detailed reviews of products, usually after a long time of interacting with them, while solving their actual problems. These reviews are not just summary assessments or recommendations, but also self-reports of the user's experience. Research shows that 49\% of the sentences extracted from user reviews contain UUX information.  However, due to the broad amount of reviews and their lack of  structure, it would be very inefficient to manually analyze them and extract relevant information. This underlines the clear need for an automated solution.

\noindent
In this thesis we propose an approach to automatically evaluate UUX information present in user reviews through the following steps: (a) extract usability and user experience information from user reviews using machine learning techniques, (b) evaluate the sentiment of each review sentence using sentiment analysis and (c) visualize the results in different perspectives to aid the detection of UUX problems. We achieved 68\% accuracy in detecting the presence of UUX information in review sentences and 71\% accuracy in predicting the sentences sentiment. The visualization component is introduced to illustrate the potential usage scenarios of the approach, however, it is in an early stage of development. 

\noindent
The proposed approach provides a quick UUX assessment method that allows a continuous evaluation process, incorporating user feedback over time and capturing aspects that are not captured by standard UUX evaluation methods such as spontaneity of opinions.