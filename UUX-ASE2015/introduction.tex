Software quality is an important aspect of the software development process, and its absence might result in serious consequences such as financial and reputation loss \cite{Hussain2013}.
Software quality can be measured through conformance to functional requirements and other non-functional specifications defined by user needs\cite{Chattopadhyay2013}. Non-functional or quality requirements like usability, reliability, performance and supportability are factors which lead to a successful project \cite{Bayraktaroglu2009}. Various studies \cite{Lizano2013}, \cite{ Chattopadhyay2013} have highlighted the importance of software usability and in general of user experience as factors for promoting software success. 
Both, good usability and user experience (UUX), rely on user feedback through evalu- ation rather than simply trusting the experience and expertise of the designer \cite{Quesenbery2001}.
There exists an extensive number of UUX evaluation methods, however, these methods are expensive in terms of time and human resources.  They usually consist in getting feedback from the user through observations or direct interviews, in a non-natural environment. Furthermore, the interaction is captured only for a short time, making it difficult to evaluate dimensions of usability such as learnability or memorablity. Therefore, there is a need for alternative UUX evaluation methods that can address these vulnerabilities. 
User reviews offer great potential in addressing the limitations of the classical UUX evaluation methods. Users express their opinion and sentiment about software products in review sites, social media platforms and blogs. They write detailed reviews of products, usually after a long time of interacting with them, while solving their actual problems. In addition, these reviews are not just summary assessments or recommendations, but also self-reports of their experiences as users. Research shows that user feedback contains a considerable amount of UUX information. Hedegaard and Simonsen \cite{Hedegaard2013} found that 49\% of the sentences extracted from user reviews contain UUX information. However, due to the broad amount of reviews and their lack of  structure, it would be very inefficient to manually analyze them and extract relevant information. This underlines the clear need for an automated solution.
We must note though that the potential usefulness of online reviews has some important caveats, compared to laboratory usability studies. 
Obviously, the software must already be on the market to be evaluated publicly.
Further, the reviews contain very few details about the reviewer (e.g., gender, age or preferences) or could even be fake.
This means that the evaluation of UUX by analyzing user reviews cannot be a replacement for the existing methods, rather than an addition to address their limitations and to provide useful information that facilitates the process of software evolution.